\documentclass[]{cv_french} % Add 'print' as an option into the square bracket to remove colors from this template for printing
\begin{document}

\header{eric}{perlinski}{Ingénieur en informatique} % Your name and current job title/field
%----------------------------------------------------------------------------------------
%	SIDEBAR SECTION
%----------------------------------------------------------------------------------------

\begin{aside} % In the aside, each new line forces a line break
\section{Coordonnées}
19 rue du général de Castelnau
Villers-Lès-Nancy, 54600
France
~
+33 6 25 20 37 34
~
\href{mailto:eric.perlinski@telecomnancy.net}{eric.perlinski@gmail.com}
\section{Langues étrangères}
Anglais (Toeic : 875), 
Japonais (débutant),
Allemand (débutant)
\section{Informatique}
C, C++, C\#
Java, Python, PHP,
ZeroMQ, RabbitMQ,
Assembleur, ADA 95,
SQL, SPARQL, OWL,
GNU-Prolog,
XML, CSS \& HTML,
JavaScript, asp.NET,
Oracle, MySQL, Jenkins,
Git, Mercurial, Debian
\end{aside}

%----------------------------------------------------------------------------------------
%	EDUCATION SECTION
%----------------------------------------------------------------------------------------
\section{Cursus scolaire}

\begin{entrylist}

%------------------------------------------------

\entry
{2016--2013}
{Diplôme d'ingénieur en informatique}
{Télécom Nancy, Villers-Lès-Nancy, France}
{Spécialisé en ingénierie du logiciel}

%------------------------------------------------

\entry
{2013--2011}
{DUT Informatique}
{Université de Lorraine, Metz, France}
{Diplôme universitaire de technologie, délivré après deux années dans l'enseignement supérieur}

%------------------------------------------------

\end{entrylist}

%----------------------------------------------------------------------------------------
%	WORK EXPERIENCE SECTION
%----------------------------------------------------------------------------------------

\section{Expérience professionnelle}

\subsection{Stages}

\begin{entrylist}

%------------------------------------------------

\entry
{2015 (2 mois)}
{Starburst Computing}
{Loria, Vandoeuvre-Lès-Nancy, France}
{\emph{Stagiaire développeur} \\
Un adaptateur écrit en C++ a été créé entre Starburst Engine et un moteur de jeu vidéo. Un jeu vidéo complet a également été développé sous Unreal Engine 4 afin de pouvoir tester l'adaptateur. 
}

\entry
{2013 (2 mois)}
{CoPROcess S.A.}
{Luxembourg-ville, Luxembourg}
{\emph{Stagiaire développeur web} \\
CoPROcess S.A. est une entreprise d'audit financier et de consulting. J'ai été assigné à la création d'un site web pour l'ESM-IAE, une école de management. Ce site web devait permettre à l'école d'avoir une plateforme de communication pour l'association des anciens élèves.
}

%------------------------------------------------

\end{entrylist}

\subsection{Projets scolaires}

\begin{entrylist}

\entry
{2015 - 2016}
{Projet industriel : Création d'une intelligence artificielle}
{NGHS, Versailles, France}
{ Projet industriel d'une durée de 5 mois qui concerne la mise en place d'une intelligence artificielle au sein d'un jeu vidéo. Le projet s'inscrit dans le cadre des projets industriels de Télécom Nancy en collaboration avec des entreprises. }

\entry
{2015}
{Projet de recherche : Intelligence artificielle distribuée}
{Loria, Vandoeuvre-Lès-Nancy, France}
{ Travaux de recherche concernant la distribution en réseaux des calculs d'une intelligence artificielle. Une architecture client-serveur écrite en C++ avait été mise en place et permettait de transférer des données à partir d'un client sous Unity3D.}

\entry
{2015}
{Création d'un compilateur}
{Télécom Nancy, Villers-Lès-Nancy, France}
{ Création d'un compilateur en java vers de l'assembleur. Le projet  permet de compiler du code d'un langage défini par l'enseignant vers de l'assembleur. }

\entry
{2015}
{Création d'un site web communautaire}
{Télécom Nancy, Villers-Lès-Nancy, France}
{ Site web développé en python / html / JavaScript. Créé à partir de la plateforme Google App Engine. }

%------------------------------------------------

\end{entrylist}


%----------------------------------------------------------------------------------------
%	INTERESTS SECTION
%----------------------------------------------------------------------------------------

\section{Intérêts}

\textbf{professionnels:} ingénierie du logiciel, développement d'applications, calculs distribués, intelligence artificielle. \\
\textbf{personnels:} actualités, musique, cinéma, jeux vidéos, films d'animations japonais.


%----------------------------------------------------------------------------------------

\end{document}